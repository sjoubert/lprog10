\documentclass[a4paper, 11pt]{article}

\usepackage[utf8]{inputenc} % Encodage des sources
\usepackage[frenchb]{babel} % Typographie francaise
\usepackage[T1]{fontenc} % Pour la police
\usepackage{lmodern} % Pour la police
\usepackage{titling} % Pour récupérer les info de la page de titre
\usepackage[plain]{fullpage} % Utilisation de toute la page

%% Personnalisation du pied de page et de l'en-tete %%
\usepackage{fancyhdr}
\pagestyle{fancy}
\lhead{}
\chead{}
\rhead{}
\renewcommand{\headrulewidth}{0pt}
\lfoot{\thetitle}
\cfoot{}
\rfoot{\thepage}
\renewcommand{\footrulewidth}{0.4pt}

%% Parametre du titre perso %%
\title{PVETE - lprog10\\
Vers une carte d'identité des langages de programmation}
\author{{\sc Joubert} Sylvain \hspace{1cm} {\sc Mignot} Philippe}
\date{19 Novembre 2010}

\begin{document} %%%%%%%%%%%%%%%%%%%%%%%%%%%%%%%%%%%%%%%%%%%%%%%%%%%%%%%%

%% Titre %%
\begin{center}
	\LARGE \thetitle \normalsize \\
	\vspace{2\baselineskip}
	\thedate \\
	\vspace{\baselineskip}
	\theauthor
\end{center}
\vspace*{5em}

\section{Problématique}

\section{Plan d'attaque}
Nous pensons procéder comme suit :

\begin{enumerate}
\item Cadrer plus précisément le sujet. Préciser notamment la notion de langage de programmation (le HTML ou le XML en font-ils partie ?, …)
\item Collecter toute une liste de caractéristiques et de notions liées aux langages de programmation
\item Étudier et comprendre ces notions
\item Les classer par thèmes afin d'aboutir au squelette de notre carte d'identité
\item Présenter quelques exemples de langages via leur carte d'identité
\end{enumerate}

\end{document}