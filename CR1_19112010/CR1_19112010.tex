\documentclass[a4paper, 11pt]{article}

\usepackage[utf8]{inputenc} % Encodage des sources
\usepackage[frenchb]{babel} % Typographie francaise
\usepackage[T1]{fontenc} % Pour la police
\usepackage{lmodern} % Pour la police
\usepackage{titling} % Pour récupérer les info de la page de titre
\usepackage[plain]{fullpage} % Utilisation de toute la page

%% Personnalisation du pied de page et de l'en-tete %%
\usepackage{fancyhdr}
\pagestyle{fancy}
\lhead{}
\chead{}
\rhead{}
\renewcommand{\headrulewidth}{0pt}
\lfoot{\thetitle}
\cfoot{}
\rfoot{\thepage}
\renewcommand{\footrulewidth}{0.4pt}

%% Parametre du titre perso %%
\title{PVETE - lprog10\\
Vers une carte d'identité des langages de programmation}
\author{{\sc Joubert} Sylvain \hspace{1cm} {\sc Mignot} Philippe}
\date{19 Novembre 2010}

\begin{document} %%%%%%%%%%%%%%%%%%%%%%%%%%%%%%%%%%%%%%%%%%%%%%%%%%%%%%%%

%% Titre %%
\begin{center}
	\LARGE \thetitle \normalsize \\
	\vspace{2\baselineskip}
	\thedate \\
	\vspace{\baselineskip}
	\theauthor
\end{center}
\vspace*{5em}

\section{Problématique}
Si parmi les développeurs une grande majorité sait par exemple que le Java et le C++ sont des langages objets et sont parfois capables d'en citer d'autres, un nombre beaucoup plus réduit est capable de citer des langages réflexifs, fonctionnels, à programmation par contrat, …

Et pour cause, il faut déjà savoir de quoi il s'agit.\\

C'est entre autre dans cette optique que se situe ce projet de veille technologique : décrire ce qu'est un langage de programmation via ses caractéristiques.

Le but final est de construire ce que l'on pourrait appeler une carte d'identité des langages de programmation, à savoir un squelette regroupant des caractéristiques (techniques notamment) comme, par exemple, le type du code objet (s'agit-il de binaire, de byte-code ou directement des sources).

Chaque langage est ainsi spécifique de par son agrégation de notions. Il s'agit donc de décrypter l'ADN des langages de programmation.\\

De plus, même si ce n'est pas l'objectif de ce projet, une fois cette carte d'identité construite cela pourrait permettre de générer une base de donnée des langages et d'y effectuer des recherches plus simplement et efficacement.

En effet, comment feriez-vous aujourd'hui pour savoir quels sont les langages objets, réflexifs, compilés et supportant la généricité du code ?
	
\section{Plan d'attaque}
Nous pensons procéder comme suit :

\begin{enumerate}
\item Cadrer plus précisément le sujet. Préciser notamment la notion de langage de programmation (le HTML ou le XML en font-ils partie ?, …), de paradigme de programmation, …
\item Collecter toute une liste de caractéristiques et de notions liées aux langages de programmation
\item Étudier et comprendre ces notions
\item Les classer par thèmes afin d'aboutir au squelette de notre carte d'identité
\item Présenter quelques exemples de langages via leur carte d'identité
\end{enumerate}

\end{document}