\section{Cartes d'identités}

\subsection{Python}

\begin{figure}[!ht]
	\center
	\includegraphics[scale=0.4]{img/python.jpeg}
\end{figure}

\renewcommand{\labelitemi}{\textbullet}
\begin{itemize}
\item Généralités
	\begin{itemize}
	\item nom : Python
	\item version : 3.2
	\item date de création : 1990
	\item créateur : Guido van Rossum\\
	\end{itemize}
\item Paradigmes
	\begin{itemize}
	\item impératif
	\item orienté objet
	\item fonctionnel
	\item réflexif\\
	\end{itemize}
\item Typage
	\begin{itemize}
	\item fort
	\item dynamique
	\item implicite\\
	\end{itemize}
\item Divers
	\begin{itemize}
	\item turing-complétude : oui
	\item évaluation : stricte
	\item gestion de la mémoire : à la charge de l'implémentation\\
	\end{itemize}
\item Communauté
	\begin{itemize}
	\item popularité : langage populaire, parmi les 10 premiers
	\item implémentations : CPython, Jython, IronPython, PyPy\\
	\end{itemize}
\item Code
	\begin{itemize}
	\item code objet : principalement le code source mais du byte code peut aussi être généré
	\item Hello world! :
\begin{lstlisting}[language=python]
# Affichage standard
print("Hello world!")
\end{lstlisting}
	\end{itemize}
\end{itemize}