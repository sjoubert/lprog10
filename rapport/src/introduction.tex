\section{Introduction}

\subsection{Problématique}
Si parmi les développeurs une grande majorité sait par exemple que le Java et le C++ sont des langages objets et sont parfois capables d'en citer d'autres, un nombre beaucoup plus réduit est capable de citer des langages réflexifs, fonctionnels, à programmation par contrat, …
Et pour cause, il faut déjà savoir de quoi il s'agit.\\

C'est entre autre dans cette optique que se situe ce projet de veille technologique : décrire ce qu'est un langage de programmation via ses caractéristiques.
Le but final est de construire ce que l'on pourrait appeler une carte d'identité des langages de programmation, à savoir un squelette regroupant des caractéristiques (techniques notamment) comme, par exemple, le type du code objet (s'agit-il de binaire, de byte-code ou directement des sources).
Chaque langage est ainsi spécifique de par son agrégation de notions. Il s'agit donc de décrypter l'ADN des langages de programmation.\\

De plus, même si ce n'est pas l'objectif de ce projet, une fois cette carte d'identité construite cela pourrait permettre de générer une base de donnée des langages et d'y effectuer des recherches plus simplement et efficacement.
En effet, comment feriez-vous aujourd'hui pour savoir quels sont les langages objets, réflexifs, compilés et supportant la généricité du code ?

\subsection{Contexte}

\subsubsection{Langage informatique}

Un langage est constitué d’un ensemble de signes doté d’une sémantique, et le plus souvent d’une syntaxe. Contrairement aux langages naturels, qui sont destinés à la communication entre les humains, les langages informatiques ont pour but de décrire l’ensemble des actions consécutives qu’une machine doit effectuer. Les protocoles de communication, qui servent à la communication entre les machines, ne sont donc pas considérés comme des langages informatiques.\\

Un langage informatique a pour but originel la conception, la mise en œuvre ou l’exploitation d’un système d’information. Néanmoins, ils sont également utilisés pour effectuer d’autres tâches, comme la maintenance du système d’information, la création d’interface graphique, la programmation de commande numérique, la statistique, etc. Les langages de conception (UML, Merise, etc.), les langages de structuration de contenu (HTML, XML, tous les langages de balisage en général), les méthodes formelles, les langages de définition de données et les langages de requêtes (SQL, etc.) sont des langages informatiques, ainsi que, bien sûr, les langages de programmation.\\

Ce type de langage est destiné à la fois à la machine et à l’humain. En effet, les instructions contenues dans le code correspondent à des actions que doit effectuer le processeur. \cite{bib_infoclick} En outre, le code doit pouvoir être écrit et compris par les humains. Le langage utilisé par le processeur est le langage machine, constitué de données binaires. Pour que le processeur exécute les instructions définies dans le code du langage informatique, il faut d’abord le transformer en langage machine.

\subsubsection{Langage de programmation}

Un langage de programmation fournit une abstraction de niveau supérieur pour utiliser une machine, car très peu d’humains comprennent le langage machine. La notion de programme est apparue au cours de la deuxième moitié du XIXème siècle, mais les premiers langages de programmation datent des environs de l’année 1950. Dans les années 60, de nombreux langages spécialisés sont créés. Les langages dits généraux commencent à s’imposer à partir des années 70 (langage C et Pascal). Les langages orientés objet apparaissent dans les années 80. Vers 1990 arrivent les langages Java, Perl et Python. \cite{bib_knuth} La programmation Internet apparaît dans les années 2000.\\

Étant donné que tout le monde peut créer son propre langage de programmation, il est impossible d’estimer le nombre total de langages de programmation avec précision, mais il est sûr que ce nombre est de l’ordre de quelques milliers. \cite{bib_scriptol} \cite{bib_dowek} La transformation du code source d’un langage de programmation en langage machine peut se faire de différentes façons. En premier lieu, le code peut être interprété par un logiciel interpréteur, qui va lire chaque ligne du code au cours de l’exécution du programme, et envoyer l’instruction correspondante au processeur sous forme binaire. En second lieu, le code peut être transformé directement sous forme exécutable par un compilateur. Cette solution permet une meilleure performance, parce que le programme est compilé une bonne fois pour toutes. Cependant, elle nécessite de recompiler le code source à chaque modification. Elle garantit également la sécurité du code source, car le programme ainsi diffusé est en langage machine.\\

Une fois défini ce que l’on entend exactement par langage de programmation on se permettra d’utiliser l’expression «langage» seule pour «langage de programmation».

\subsubsection{Organisation de la suite de l'étude}

La suite de ce rapport sera organisée de la façon suivante :
\begin{itemize}
\item résultat de l'étude : une carte d'identité des langages de programmation,
\item présentation des différentes caractéristiques de la carte d'identité,
\item proposition de quelques exemples.\\
\end{itemize}
